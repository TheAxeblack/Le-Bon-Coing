\documentclass[11pt,a4paper]{article}

\usepackage[top=3cm, bottom=3cm, left=2.5cm, right=2.5cm]{geometry}
\usepackage{amsmath}
\usepackage{amssymb}
\usepackage{graphicx}
\usepackage{psfrag}
\usepackage[utf8]{inputenc}
\usepackage[T1]{fontenc}
\usepackage[french]{babel}
\usepackage[french,vlined,boxed]{algorithm2e}
\usepackage{marvosym}
\usepackage{dsfont}
\usepackage{hyperref}

\pagestyle{headings}

\title{Développement Web\\Rapport de projet \textbf{Le Bon Coing}}
\author{Lina BENALLI\\Soheil BENABIDA\\Mathis LÉCUYER}
\date{}

\begin{document}

\maketitle

\newpage

\tableofcontents

\newpage

\section{Présentation du projet}
Dans le cadre de notre formation, nous avions a réaliser un projet dans l'unité d'enseignement de \textbf{Développement Web}.\\
Le but de ce projet était de manipuler les quatres langages rencontré durant le cours dans la réalisation d'un site web à savoir \emph{HTML}, \emph{CSS}, \emph{JavaScript} et \emph{PHP}.\\
Le choix du thème étant libre nous sommes donc parti sur un site de petites annonces du type \underline{Leboncoin} que nous avons nommé \underline{\textbf{Le Bon Coing}}.\\
Nous avions un certains nombre de contraintes techniques à respecter durant la réalisation de ce projet. 
\begin{itemize}
    \item création d'un site structuré et valide (balises HTML et après validation sur le site de \href{https://validator.w3.org}{W3C})
    \item utilisation d'une feuille de style CSS indépendante validée par \href{https://jigsaw.w3.org/css-validator/}{W3C}
    \item mise en place de quelques fonctionnalitées en JavaScript
    \item utilisation du PHP et des sessions
    \item gestion d'une base de données et utilisation de PDO
    \item le site devra fonctionner sur webetu (plateforme d'hébergement de l'université)
\end{itemize}

\textbf{Le Bon Coing} est un marché en ligne permettant des personnes de tous types et de tous horizons d'entrer en contact afin de réaliser une ou plusieurs transactions.\\
Ainsi sur \textbf{Le Bon Coing}, vous pourrez découvrir plusieurs types annonces. Allant de la voiture à rénover, à la dernière console de jeux à la mode en passant par des confitures faites avec amour.

\section{Répartition des tâches}
Lors de ce projet et après concertation entre nous, nous nous sommes réparti les tâches en fonction des affinitées de chacun avec les différents langages.
\begin{center}
    \begin{tabular}{| c | c | c | }
      \hline
      Lina & Soheil & Mathis \\ \hline
      Conception de la Base de donnée & Réalisation du logo & Réalisation du wireframe \\
      Insertion des annonces & Insertion des utilisateurs & Réflexion sur l'ergonomie \\
      Vérification des données & Affichage des annonces & Développement des fonctionnalitées \\
      Vérification des pages sur W3C & Gestion et suppression des annonces & Inscription/Connexion et Design \\
      \hline
      \multicolumn{3}{|c|}{Rédaction du Rapport} \\
      \hline
    \end{tabular}
  \end{center}


\section{Structure du site}
\subsection{Les pages}
\textbf{Le Bon Coing} est composé de sept pages accessibles différentes et de quatre pages de script PHP. Pour l'ensemble du site, nous avons 4 fichiers .CSS associé aux pages .PHP
\begin{itemize}
    \item Les pages accessibles
    \begin{enumerate}
        \item \emph{home.php} qui est la page d'accueil du site web. Elle contient une section répertoriant les cinq dernières annonces mises en lignes par les utilisateurs ainsi qu'un formulaire de recherche afin de trouver une annonce succeptible de plaire à l'utilisateur.
        \item \emph{connexion.php} permettant à un utilisateur de se connecter au site. Si l'utilisateur essayant de se connecter n'est pas enregistrer un lien permet de le renvoyer sur la page \emph{inscription.php}
        \item \emph{inscription.php} permettant comme son nom l'indique de s'inscrire sur le site. Chaque informations entrée sur cette page sera ensuite enregistrée dans la base de donnée du site.
        \item \emph{deposer-annonce.php} est sans doute l'une des pages les plus importantes avec \emph{gestion.php}. En effet, elle permet à l'utilisateur de déposer une annonce sur le site.\\ Cette dernière sera ensuite enregistrée dans la base de données.
        \item \emph{gestion.php} permet à un utilisateur inscrit et connecté au site de gérer les options de son compte mais également les annonces qu'il a mise en ligne.
        \item \emph{annonce.php} qui est la page type de l'affichage des annonces. Le navigateur va récupérer les informations de l'annonce en question dans la base de données afin de pouvoir les charger dans la page.
        \item \emph{source.html} qui contient les sources des images libres de droit utilisées
    \end{enumerate}
    \item Les pages de script PHP
    \begin{enumerate}
        \item \emph{deconnexion.php} qui deconnecte l'utilisateur et supprime la session PHP créer lors de la connexion
        \item \emph{supprimer.php} qui supprime l'annonce choisie
        \item \emph{modifier.php} qui modifie l'annonce sélectionnée
        \item \emph{modifier-profil.php} qui modifie les informations du profil de l'utilisateur
    \end{enumerate}
\end{itemize}

\subsection{Les fonctionnalitées mises en place}
Afin de proposer de l'interaction à l'utilisateur, nous avons implémenté quelques fonctionnalitées JavaScript.\\
A savoir, pour l'ensemble du site~:~un menu déroulant, un affichage de formulaire et un mode nuit. Ainsi que pour la page des annonces un carousel d'images.\\
Une autre fonctionnalitée, la traduction du site a commencé à être développée, cependant du à un manque de temps, elle est incomplete.\\
La fonctionnalitée de "Sticky navbar" a été abandonnée en cours de route car cette dernière entrainait des erreurs.

\subsection{La base de données}
La base de données du site est une \emph{base de donnée SQlite}, elle est composée de deux tables.
\begin{enumerate}
    \item La table \emph{annonce\_p}, il s'agit de la table liée aux annonces des différents utilisateurs.\\
    Pour la modélisation d'une annonce nous avons décidé de la définir par
    \begin{itemize}
        \item \textbf{un identifiant} unique à chaque annonce
        \item \textbf{un identifiant d'utilisateur} qui est directement associé à l'utilisateur en question, il doit être unique.
        \item \textbf{un nom} afin de préciser à l'utilisateur ce que représente l'annonce.
        \item \textbf{un type} permettant de classer les différentes annonces en fonction de ce que l'utilisateur recherche.
        \item \textbf{une date d'ajout} afin de savoir quand l'annonce a été ajouté sur le site.
        \item \textbf{des images} permettant à l'utilisateur d'avoir une visibilité sur ce qui proposé.
        \item \textbf{une description} afin de donner plus d'informations à la personne consultant l'annonce (quelles sont les raisons de la mise en vente, anciennetée, etc\dots)
        \item \textbf{un prix}
        \item \textbf{un code postal} pour informer l'utilisateur de la localisation du bien présenté dans l'annonce.
    \end{itemize}
    \item La table \emph{user}, il s'agit de la table liée aux utilisateurs du site.\\
    Les utilisateurs sont définis par 
    \begin{itemize}
        \item \textbf{un identifiant} unique à chaque utilisateur
        \item \textbf{un genre} 
        \item \textbf{un prénom}
        \item \textbf{un nom} 
        \item \textbf{un pseudo}
        \item \textbf{un mot de passe} crypté en MD5 (Message Digest 5)
        \item \textbf{un email} afin de pouvoir envoyer un email
        \item \textbf{un statut} afin de savoir si l'utilisateur est un administrateur ou pas.
    \end{itemize}
\end{enumerate}

\end{document}
