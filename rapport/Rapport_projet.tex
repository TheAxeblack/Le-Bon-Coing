\documentclass[11pt,a4paper]{article}

\usepackage[top=3cm, bottom=3cm, left=2.5cm, right=2.5cm]{geometry}
\usepackage{amsmath}
\usepackage{amssymb}
\usepackage{graphicx}
\usepackage{psfrag}
\usepackage[utf8]{inputenc}
\usepackage[T1]{fontenc}
\usepackage[french]{babel}
\usepackage[french,vlined,boxed]{algorithm2e}
\usepackage{marvosym}
\usepackage{dsfont}
\usepackage{hyperref}

\pagestyle{headings}

\title{Développement Web\\Rapport de projet \textbf{Le Bon Coing}}
\author{Lina BENALLI - Soheil BENABIDA - Mathis LÉCUYER}
\date{}

\begin{document}

\maketitle

\newpage

\tableofcontents

\newpage

\section{Présentation du projet}
Dans le cadre de notre formation, nous avions a réaliser un projet dans l'unité d'enseignement de \textbf{Développement Web}.\\
Le but de ce projet était de manipuler les quatres langages rencontré durant le cours dans la réalisation d'un site web. Le choix du thème étant libre nous sommes donc parti sur un site de petites annonces du type \underline{Leboncoin}.\\
Nous avions un certains nombre de contraintes techniques à respecter durant la réalisation de ce projet.
\begin{itemize}
    \item création d'un site structuré et valide (balises HTML et après validation sur le site de \href{https://validator.w3.org}{W3C})
    \item utilisation d'une feuille de style CSS indépendante validée par \href{https://jigsaw.w3.org/css-validator/}{W3C}
    \item mise en place de quelques fonctionnalitées en JavaScript
    \item utilisation du PHP et des sessions
    \item gestion d'une base de données et utilisation de PDO
    \item le site devra fonctionner sur webetu (plateforme d'hébergement de l'université)
\end{itemize}
\end{document}
