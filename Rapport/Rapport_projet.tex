\documentclass[11pt,a4paper]{article}

\usepackage[top=3cm, bottom=3cm, left=2.5cm, right=2.5cm]{geometry}
\usepackage{amsmath}
\usepackage{amssymb}
\usepackage{graphicx}
\usepackage{psfrag}
\usepackage[utf8]{inputenc}
\usepackage[T1]{fontenc}
\usepackage[french]{babel}
\usepackage[french,vlined,boxed]{algorithm2e}

\pagestyle{headings}
\setlength{\parindent}{0pt}
\title{Bases de Données\\Compte rendu du TP3}
\author{Mathis LÉCUYER}
\date{}

\begin{document}

\maketitle

\section*{Partie 2}
\subsection*{1.}
On nous demande d'écrire une fonction \textbf{NOUV\_NATION} qui permettra d'insérer, dans la base de données mise à notre disposition, un pays avec plusieurs informations le concernant. Ces informations seront rentrés en paramètres lors de l'appel de la fonction.\\
Dans un premier temps, commençont par mettre en place un delimiter afin de pouvoir définir cette fonction.\\
Si l'on ne le met pas en place la suites de requêtes SQL ne s'effectuera pas comme il faut étant donné que le delimiter par défaut de SQl est le point-virgule.
\lstinputlisting[language=sql, firstline=3, lastline=20]{../nouv-nation.sql}

Comme dit précedemment, on définit la fonction.\\
Pour cela on utilise \textbf{CREATE FUNCTION} suivit de ses différents paramètres et de leurs types. Ainsi que du type de donnée renvoyée par la fonction.\\
Dans notre cas on a :
\begin{itemize}
  \item \emph{nomP} représentant le nom du pays que l'on souhaite insérer
  \item \emph{codeP} réprésentant le code du pays
  \item \emph{capitaleP} qui représente la capitale du pays
  \item \emph{superficieP} représentant la superficie du pays
  \item \emph{populationP} réprésentant le nombre d'habitants du pays
  \item \emph{typegouvP} représentant le type de gouvernement à la tête du pays
  \item \emph{ddi} qui représente la date d'indépendance du pays
\end{itemize}

Le type renvoyé est indiqué par l'instruction \textbf{RETURNS INT}, dans notre cas ce sera un entier.\\
On a ensuite le mot-clé \textbf{BEGIN} qui marque le début de la suite d'instructions qui seront effectuées lors de l'appel de la fonction.\\
On déclare une variable locale nommée \emph{erreur} de type \textbf{INT} avec une valeur par défaut de 1.\\
On déclare ensuite une condition qui sera vérifiée en cas de duplication de clé primaire ainsi qu'un gestionnaire d'erreurs qui si la condition est vérifié met la valeur de la variable \emph{erreur} à 0.\\
Après quoi on insert dans la table \emph{country} les valeurs de \emph{nomP, codeP, capitaleP, NULL, superficieP et populationP} ainsi que dans la table \emph{politics} les valeurs de \emph{codeP, typegouvP et ddi}.\\
Et on retourne la valeur de la variable erreur, après quoi on remet le delimiter à sa valeur par défaut c'est-à-dire le point-virgule.\\
\\
Maintenant on va appeler cette fonction et insérer dans notre base de donnée le Kosovo.
\lstinputlisting[language=sql, firstline=22, lastline=23]{../nouv-nation.sql}

\begin{figure}[h]
  \centering
  \includegraphics[scale=0.5]{imgs/nouv-nation.png}
  \caption{Résultat de l'éxecution de la fonction \textbf{NOUV\_NATION}}
  \label{fig1}
\end{figure}

Pour s'assurer de la bonne éxecution de la fonction, on va effectuer une sélection de tous les champs de la table country où le code du pays correspond au code que nous avons insérer dans l'appel de la fonction c'est-à-dire le code 'XK'.

\begin{figure}[h]
  \centering
  \includegraphics[scale=0.5]{imgs/nouv-nation_confirm.png}
  \caption{Résultat de \textbf{SELECT * FROM nation WHERE Code LIKE 'XK'}}
  \label{fig2}
\end{figure}

\subsection*{2.}
Pour cette section on doit créer une nouvelle table dans notre base de donnée elle s'appelera \textbf{GDP} et aura deux champs, le champ \emph{Pays} qui représente le code du pays, et \emph{GDP} qui représente le \textbf{\underline{Produit National Brut}} du pays.

\lstinputlisting[language=sql, firstline=7, lastline=11]{../gdp_par_pays.sql}

Après quoi, on souhaite créer une fonction \textbf{CHARGE\_GDP\_CONTINENT} qui calculera la somme des GDP de tous les pays du continent.\\
On commence par définir la procédure \textbf{GDP\_PAR\_PAYS} qui sera appelé dans notre fonction, elle prend en paramètre d'entrée le code du pays \emph{codeP}, et en parametre de sortie une variable pour les erreurs \emph{erreur} et le GDP du pays \emph{gdpP}.

\lstinputlisting[language=sql, firstline=13, lastline=29]{../gdp_par_pays.sql}

On déclare une condition qui sera vérifiée en cas de duplication de clé primaire ainsi qu'un gestionnaire d'erreurs qui si la condition est vérifié met la valeur de la variable \emph{erreur} à 0.\\
On met la valeur de erreur à 1.\\
Après quoi on demande au système de gestion de base de données de faire une sélection du GDP depuis la table \emph{economy} où \emph{codeP} est égale au champ \emph{Country} de la table \emph{economy} et de stocker dans la variable \emph{gdpP} les valeurs trouvées.\\\\
On va maintenant, définir la fonction \textbf{CHARGE\_GDP\_CONTINENT} mais avant on change le delimiter afin de ne pas avoir de problème. La fonction prend en paramètre le nom du continent dont ou souhait afficher le GDP qui sera un entier.

\lstinputlisting[language=sql, firstline=30, lastline=63]{../gdp_par_pays.sql}

On commence par déclarer nos variables locales.
\begin{itemize}
	\item \emph{done} (de type \textbf{INT}) pour la gestion de la boucle que l'on verra plus tard. Sa valeur par défaut vaut 0.
	\item \emph{codeP} (de type \textbf{VARCHAR}) qui nous permettra de stocker les valeurs de country.Code
	\item \emph{gdp, erreur et somme} (de type INT) qui nous permettrons respectivement
	\begin{itemize}
		\item de stocker les valeurs de economy.GDP.
		\item d'appeler la procèdure \textbf{GDP\_PAR\_PAYS}.
		\item de faire la somme de tous les GDP.
	\end{itemize}
\end{itemize}
Puis on déclare un curseur nommé \textit{c1} qui pointe sur le résultat d'une requête ligne par ligne. Dans notre cas, la requête est la selection des champs country.Code et economy.GDP.\\
On déclare ensuite un gestionnaire d'erreur qui si l'on a plus de valeur à parcourir dans notre table mettre la valeur de \emph{done} à 1.\\
On ouvre le curseur c1, puis on dit au système de base de données de repéter les instructions qui suivent jusqu'à ce que la valeur de \emph{done} soit égale à 1.\\
Les instruction à répéter sont les suivantes.
\begin{itemize}
	\item on stocke les valeur des champs pointés par c1 dans les variables \emph{codeP} et \emph{gdpP}
	\item on appel la procèdure \textbf{GDP\_PAR\_PAYS} avec en paramètres \emph{codeP, erreur, gdpP}
	\item on fait en sorte que \emph{somme} prenne la valeur d'elle même + la valeur de la variable \emph{gdpP}
\end{itemize}
Puis on ferme le curseur c1.

\begin{figure}[h]
	\centering
	\includegraphics[scale=0.9]{imgs/gdp_africa.png}
	\caption{Résultat de \textbf{SELECT CHARGE\_GDP\_CONTINENT('Africa');}}
	\label{fig3}
\end{figure}

\begin{figure}[h]
	\centering
	\includegraphics[scale=0.9]{imgs/gdp_america.png}
	\caption{Résultat de \textbf{SELECT CHARGE\_GDP\_CONTINENT('America');}}
	\label{fig4}
\end{figure}

\begin{figure}[ht]
	\centering
	\includegraphics[scale=0.9]{imgs/gdp_asia.png}
	\caption{Résultat de \textbf{SELECT CHARGE\_GDP\_CONTINENT('Asia');}}
	\label{fig5}
\end{figure}

\begin{figure}[ht]
	\centering
	\includegraphics[scale=0.9]{imgs/gdp_australia.png}
	\caption{Résultat de \textbf{SELECT CHARGE\_GDP\_CONTINENT('Australia/Oceania');}}
	\label{fig6}
\end{figure}

\newpage

\begin{figure}[ht]
	\centering
	\includegraphics[scale=0.9]{imgs/gdp_europe.png}
	\caption{Résultat de \textbf{SELECT CHARGE\_GDP\_CONTINENT('Europe');}}
	\label{fig7}
\end{figure}

\subsection*{3.}
On souahite ajouter deux nouvelles colonnes à la table country. Une colonne AncienneCapitale et une colonne ChangerDateCapitale.\\
Pour cela on va utiliser les requêtes suivantes :

\lstinputlisting[language=sql, firstline=3, lastline=8]{../capitale.sql}

Puis on va écrire un déclencheur (trigger en anglais) qui lorsque l'on souhaite faire une mise à jour de la capitale d'un pays de la table country, mettra la capitale actuelle dans la colonne AncienneCapitale et la mise à jour que l'on souhaite faire dans la colonne Capital.

\lstinputlisting[language=sql, firstline=9, lastline=22]{../capitale.sql}

Avant de créer quoi que soit, on commence par changer le delimiter afin de ne pas avoir le séparateur par défaut. Après quoi on créer notre déclencheur nommé changerCapitale.\\
Ce sera un déclencheur qui s'activera avant la mise de la table country et qui pour chaque lignes effectuera les instructions suivantes :
\begin{itemize}
	\item Un test pour savoir si donnée actuelle du champ Capital est différente de celle que l'on souhaite mettre à la place.
	\item si la condition est vraie, alors AncienneCapitale prend la valeur de l'actuelle valeur de Capital.\\
	Et ChangerDateCapitale prend la date du jour de la requête comme valeur.
	\item sinon on arrête.
\end{itemize}

Pour vérifer si notre déclencheur marche on va tester sur plusieurs valeurs.

\lstinputlisting[language=sql, firstline=24, lastline=26]{../capitale.sql}

\begin{figure}[h]
	\centering
	\includegraphics[scale=0.3]{imgs/trigger_usa.png}
	\caption{Résultat de la commande ci-dessus}
	\label{fig8}
\end{figure}

\lstinputlisting[language=sql, firstline=28, lastline=30]{../capitale.sql}

\begin{figure}[h]
	\centering
	\includegraphics[scale=0.3]{imgs/trigger_france.png}
	\caption{Résultat de la commande ci-dessus}
	\label{fig9}
\end{figure}

On test maintenant de mettre à jour un pays qui n'existe pas.

\lstinputlisting[language=sql, firstline=32, lastline=34]{../capitale.sql}

\begin{figure}[ht]
	\centering
	\includegraphics[scale=0.7]{imgs/trigger_Dirt.png}
	\caption{Résultat de la commande ci-dessus}
	\label{fig10}
\end{figure}

\newpage

\subsection*{4.}
On souhaite désormais écrire une procédure afin d'augmenter la population d'un pays choisi d'un pourcentage entré en paramètre.

\lstinputlisting[language=sql, firstline=3, lastline=17]{../augm_pop.sql}

On créer notre procédure nommée POPULATION\_AUGM.\\
On commence par déclarer une variable nommé \emph{pop} avec une valeur par défaut de 0.\\
Ensuite, on fait une sélection de la population depuis la table country où \emph{nomP} est égal au champ Name de la table country que l'on stocke dans la varible \emph{pop}.
La variable \emph{pop} prend la valeur de $pop + (pop * x / 100)$.\\
On met à jour la table country, et on affecte à Population la valeur de pop.\\

Voilà le résultat obtenu lors de l'appel de la procédure :~

\begin{figure}[ht]
	\centering
	\includegraphics[scale=0.5]{imgs/augm_pop_france.png}
	\caption{Résultat de \textbf{CALL POPULATION\_AUGM(5, 'France')}}
	\label{fig11}
\end{figure}

\subsection*{5.}
On veut maintenant calculer, la population totale des continents. Pour cela, on a va créer une procédure que l'on appelera POP\_PAR\_CONTINENT. Elle n'aura aucun paramètre d'entrée ou de sortie.

\lstinputlisting[language=sql, firstline=3, lastline=17]{../pop_continent.sql}

On demande au système de gestion de base de données de selectionner toutes les données du champ Name de la table continent que l'on affichera sous le nom "Continent", ainsi que la somme de toute les données du champ Population de la table country multipliées par les valeurs des données du champ Percentage de la table encompasses et le tout divisé par 100. On affichera cette somme sous le nom "Population totale". Cependant, on ne souhaite pas avoir toutes les valeurs, juste celles ou les valeurs du champ Name de la table continent sont présente dans le champs continent de la table encompasses et où les valeurs du champs Country de la table encompasses sont également présente dans le champ Code de la table country.\\
On groupe le tout par ordre alphabétiques des valeurs présente dans le champ Name de la table continent.

\begin{figure}[ht]
	\centering
	\includegraphics[scale=0.5]{imgs/Pop_totale_continent.png}
	\caption{Résultat de \textbf{CALL POP\_PAR\_CONTINENT();}}
	\label{fig12}
\end{figure}

\subsection*{6.}
On s'intéresse ici, aux deux langues les plus parlées de notre base de données. Comme précédemment on va écrire une procédure que l'on appelera DEUX\_LANGUES.\\

\lstinputlisting[language=sql, firstline=3, lastline=36]{../langage.sql}

On commence par définir nos variables
\begin{itemize}
	\item pop (de type INT) qui nous servira à stocker le résultat de la somme des données de Percentage multipliées par les données de Population le tout divisé par 100.
	\item erreur (de type INT avec une valeur par défaut de 1) qui nous servira à arrêter la procédure en cas d'erreur.
	\item langue1 (de type VARCHAR) qui nous servira à stocker la langue parlée.
\end{itemize}
On déclare ensuite notre curseur c2 qui pointera sur la requête suivante :~la sélection des langues que l'on appelera "Langue" et la somme des données de Percentage multipliées par les données de Population le tout divisé par 100 que l'on appelera "Somme" depuis les tables languages et country où les données du champ Code de la table country sont présente aux données du champ Country de la table language.\\
On groupe les valeurs obtenu par ordre alphabétique des langues puis on les ordonnent la valeur décroissante de Somme.\\
On se limite à deux résultats afin d'avoir seulement les deux langues les plus parlées.\\
\\
On déclare ensuite un gestionnaire d'erreur qui s'activera si on n'a pas trouvé de valeurs et on mettra la valeur de erreur à 0.\\
\\
On ouvre le curseur c2.\\
Et tant que (while en anglais) la valeur de erreur est différente de 0 faire :~
\begin{itemize}
	\item on stocke les valeurs de c2 dans les variables langue1 et pop.
	\item on demande au sytème de base de données de selectionner les valeurs de langue1, pop, country.Name et country.Population où les valeurs des données du champ Country de la table language sont également présente dans le champ Code de la table country et où les valeurs du champ Name de la table language sont égales à la valeur présente dans la variable langue1.
\end{itemize}
On sort du tant que, on ferme le curseur c2 et notre procédure est finie.\\
\\
Observons si cette procédure marche, pour cela on va appeler cette procédure avec cette commande :~

\lstinputlisting[language=sql, firstline=37, lastline=37]{../langage.sql}

Voici le résultat de l'exécution de cette procédure.

\begin{figure}[ht]
	\centering
	\includegraphics[scale=0.5]{imgs/langues.png}
	\caption{Résultat de \textbf{CALL DEUX\_LANGUES();}}
	\label{fig13}
\end{figure}

\end{document}
